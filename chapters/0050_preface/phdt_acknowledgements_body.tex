\vspace{-1.65cm}
\begin{flushright}
    \fbox{\includegraphics[width=0.35\textwidth]{ack_pal_C.png}}
\end{flushright}

\noindent Some concise acknowledgements paragraphs may not be enough to convey how this journey, as any other doctoral project, would not have been possible without the support, scientific and not, of many people.

Above all, I thank my supervisor Prof.~Carla Braitenberg, for her precious guidance through these years.
The foundations of this project stemmed from an enthusiastic chat on the potential of GOCE-derived gravity for geothermal applications, which took place at the end of April~2015: back then, I was just enquiring for a master thesis topic.
Trying to enquire the temperature field meant getting out of the `comfort zone' of gravity and delving deep into thermal modelling -- an exciting journey between geoscience specialities.
My utmostly sincere gratitude for giving me this opportunity.

I ought to extend my gratitude to the all Tectonophysics and Geodynamics section, first and foremost to Dr.~Ildiko‘ Nagy, to whom I wish the best for her retirement.
I would then like to thank Dr. Magdala Tesauro, for feedback on the thermal segment of this project and on a preliminary manuscript.
Yet as importantly, I thank all the lab mates in this wing of \textit{Palazzina~C}, for all the opportunities for discussion, prolific arguing, and mutual advice through the last years, both in the office and remotely, when scattered abroad: Francesca, Tommaso, Federico, Luigi, Marco, and anyone I may have forgotten.
I may have learned as much from `collateral collaborations' as I have from this thesis' work, spacing from tidal analysis to Central African basins in the timespan of a cup of coffee.

\bigskip

Spending a year in München has been a fundamental milestone and an unprecedented experience for me.
Although the work carried out there was of utmost importance in shaping this final work, I should not forget the sheer extent of possible opportunities I did not explore while still there.
%Surely I must regret not making the best out of such an opportunity to become proficient in German \dots
My sincere thanks go to Prof.~Roland Pail, for his availability, and to the rest of the academic staff and colleagues at the Institute, for the pleasant working atmosphere throughout all my stay, from the frantic room search in the first week up to what had then become a familiar setting and routine.

\bigskip

This journey took me through new, exciting, challenges, with level of professional independence (and consequent responsibilities) which I had never encountered before.
I may say it has been stressful and rewarding in equal measure.
I cannot take all the merit for that, it is not a project one can carry alone: non posso ignorare il naturale supporto e l'infinita pazienza dei miei genitori in tutti questi anni.
Nemmeno va tralasciato il fondamentale e invisibile contributo dei miei più cari amici.

\bigskip

The {ESF}-funded {EUSALP}-area bound fellowship also meant frequent reporting, a good deal of wading through paperwork, and all the uncertainty that results from being among the first recipients of what was then a new founding source.
Thus my gratitude also goes also to Valentina Demontis, UniTS Doctoral Registrar's Office, for her help in dealing with my countless questions and forwarding those that were still unclear to the relevant funder's offices.

\bigskip

My gratitude to Prof.~Mehdi Eshagh and Dr.~Roman Pašteka, referees to this final dissertation, for their availability and advice.
I would also like to thank Dr.~Derrick Hasterok and an anonymous reviewer for providing fruitful comments on a manuscript, which greatly improved its quality and have been an invaluable feedback, especially on the project segments I was less familiar with.

\newpage
\paragraph{Funding}
Financial support was provided by Region Friuli Venezia Giulia (Italy) through an European Social Fund co-funded 3-year PhD fellowship ({HEaD}~{FP1687011001}), under the program `{{PO}~Friuli Venezia Giulia - Fondo Sociale Europeo 2014/2020}'.
The same funding source also supported a 12-month visiting period at the Institute of Astronomical and Physical Geodesy (IAPG) of the Technical University of Munich, under the supervision of Prof.~Roland Pail, which I gratefully thank for his availability.
Financial support for the abroad period was also supplemented by an International Mobility Scholarship, from resources of the University of Trieste.

\paragraph{Data, software, computational resources}
All the maps were drawn with the Generic Mapping Tools \parencite[GMT,][]{GMT2013} and the {GMT/Matlab} toolbox \parencite{GMTmex2017}.
As thoroughly further in this thesis, the gravity forward modelling relied both on {SHTOOLS}~\parencite{Wieczorek2018} and {Tesseroids}~\parencite{Uieda2016}.
Part of the computations involved in this research benefited of resources provided by the Leibniz Supercomputing Centre (www.lrz.de) and conceded by {IAPG} during my stay there, for which I express my gratitude.
I acknowledge and thank all the producers of data cited in this work.
