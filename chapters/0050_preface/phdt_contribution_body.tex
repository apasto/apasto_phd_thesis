\subsection*{Original contribution}
The gravity-geothermal integrated technique described and applied in chapter~\ref{c:ThermAppl} is based on the thermal model described in the supplementary chapter~\ref{c:ThermModel} and shares part of the strategies devised for the gravity data reduction strategy, presented in chapter~\ref{c:SigIs}.
A first proof-of-concept with a 1D analytical model was presented at the 2016 ESA Living Planet Symposium \parencite{Pastorutti2016LPS}, and a set of results was the subject of an oral communication I gave at the 2017 EGU General Assembly \parencite{Pastorutti2017_EGUoral}.

Those first tests, which were applied along a south-north transect from the Western Alps to the Upper Rhine Graben, were then extended to the more complex thermal numerical modelling scheme described in this dissertation.
Afterwards, the iterative radioactive heat production fitting strategy was devised, its details being presented as a poster at 2018 EGU General Assembly \parencite{Pastorutti2018_EGUposter}.

The integrated application, in the form presented in this chapter~\ref{c:ThermAppl}, was further enhanced with an in depth discussion of its results and of the rationale for the choice of parameters.
It also benefited of a revised gravity data reduction scheme, developed in parallel with the experiments of chapter~\ref{c:SigIs} and exploiting its results (albeit in a mixed spatial-spectral, regional-global scheme, as described in section~\ref{s:Appl:Grav}).
This application, edited in a suitable manuscript form, was submitted to Geophysical Journal International on February 26, 2019.

The paper: \citetitle{Pastorutti2019} \parencite{Pastorutti2019} has been accepted for publication in Geophysical Journal International on July 25, 2019, following peer review.
The paper \textit{version of record} is available online at doi:~\href{https://doi.org/10.1093/gji/ggz344}{10.1093/gji/ggz344}.

As mentioned in the `Data Availability' statement of that paper, the data sets of the results are available as an open data repository (doi:\href{https://doi.org/10.5281/zenodo.3358901}{10.5281/zenodo.3358901}).
That repository includes: a volume of temperature, pressure, radioactive heat production, thermal conductivity, and density of the lithosphere in the study area; a Matlab/Octave function to extract 2D sections and 1D vertical plots of physical properties from the volume (as shown in Fig.~\ref{fig:AAsection} and \ref{fig:AAsectionCols}); gravity input data and applied reductions; and the inverted Moho depth.

The gravity data reduction / signal isolation scheme and the uncertainty propagation experiments, presented in chapter~\ref{c:SigIs}, was developed during a 12-month visiting period at TU~München, at the Ingenieurinstitut für Astronomische und Physikalische Geodäsie (IAPG), under the supervision of Prof.~Roland Pail.
It has been the subject of an oral presentation at 2019 EGU General Assembly \parencite{Pastorutti2019EGU} and two posters \parencites{Pastorutti2019LPS}{Pastorutti2019IUGG}.
