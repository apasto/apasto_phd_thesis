This doctoral dissertation is aimed at obtaining regional surface heat flow estimates on stable continental lithosphere from satellite-derived global models of the Earth static gravity field, using variations in crustal thickness as a gravity-derived thermal constraint. The heat transported across the Earth surface, arising from its cooling interiors, senses a wide assortment of phenomena, from near-surface paleoclimatic perturbations to the driving forces of plate tectonics and mantle dynamics. Surface heat flow (SHF) is sensed with borehole measurements of temperature and conductivity - this implies that it is a costly observable, highly depending on logistic and financial constraints, on previous exploration interest and on availability of data. Different components superimpose on the SHF signal: over continents, at regional scales, most of the static signal can be explained by a partition between a basal flow through the crust-mantle boundary and the contribution of radioactive heat production (RHP). The generation of continental crust through magmatic differentiation significantly enriched it with radioactive elements. The indirect estimation of their concentration is a non trivial task, since it does not directly affect petrophysical quantities by a significant amount. However, other processes concur in affecting sensible quantities: empirical relationships from velocity and density to heat production have developed and tested satisfactorily. In addition, multi-observable modelling is regularly employed to improve the understanding on the subsurface distribution of temperature and thermal parameters. The gravity field is particularly sensitive to variations in crustal thickness, arising from the significant density contrast at the Moho. Compared with the problematically sampled SHF, gravity data benefits from a more uniform coverage. While terrestrial data still suffers from data availability issues, models derived from satellite gravimetry missions provide an unprecedented spatial homogeneity. The global gravity models derived from the European Space Agency GOCE mission provide a ppm-level accuracy for g with a resolution of about 70 km at the Earth surface. These models have already shown promising results in global crustal thickness modelling. This basis provided the motivation to assess how satellite gravity data could contribute to thermal modelling. I devise and test a strategy to constrain the crustal RHP using a gravity-derived Moho and the available heat flow data, with the aim of overcoming the issues with SHF interpolation over no-data areas. The strategy is supported by a 3D finite-difference based thermal forward model, developed ad-hoc for the project, and a gravity data reduction strategy, to isolate the Moho undulation signal. It is supported by a global reduction modelling scheme which includes a Monte Carlo error propagation of model uncertainties, complemented by a series of validation tests against spatial-domain forward modelling of test mass distributions. The tested strategy, albeit relying on a simplified uniform model of error distribution, proved fit for purpose at the target resolution (maximum degree and order of 300, aimed at data reduction of satellite-only models) and may be easily scaled. An application test in Central-Eastern Europe is presented in the last chapter. It integrates data from a satellite-only GGM, reductions with a mixed spatial- and spectral-domain modelling scheme, and heat flow measurements. It resulted in a 3D lithospheric model of temperature and thermal parameters, fitting available data and providing added information respect to the structure of the lithosphere. The modelled thermal regime is coherent with what geological inference expects and with other models of the area, keeping into account the limitations of a method which relies on as little a-priori data as possible, by design. The gravity-derived Moho is benchmarked against existing crustal models in the area.
