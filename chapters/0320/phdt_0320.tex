% input this at the beginning of each chapter-draft
\documentclass[12pt,a4paper]{scrbook}

% encoding, ams, microtype, graphicx
\usepackage[utf8]{inputenc}
\usepackage[T1]{fontenc}
\usepackage{amsmath,amsfonts,amsthm,amssymb,bm}
\usepackage[protrusion=true,expansion=true]{microtype}
\usepackage[dvips]{graphicx}
\usepackage{mathtools} % underbrace

% % fbb font (from documentation)
% \usepackage[full]{textcomp}          % to get the right copyright, etc.
% \usepackage[lining,tabular]{fbb}     % so math uses tabular lining figures
% \usepackage[scaled=.95,type1]{cabin} % sans serif in style of Gill Sans
% \usepackage[varqu,varl]{zi4}         % inconsolata typewriter
% \usepackage[libertine,bigdelims]{newtxmath}
% \usepackage[cal=boondoxo,bb=boondox,frak=boondox]{mathalfa}
% \useosf % change normal text to use proportional oldstyle figures
% % \usetosf would provide tabular oldstyle figures in text

\graphicspath{{../../graphics/}}
\usepackage[percent]{overpic}

\usepackage{newtxtext}

\usepackage{enumitem}
\usepackage{booktabs}
\usepackage{multirow}
\usepackage{bigdelim}
\usepackage{adjustbox}
\usepackage[mathlines]{lineno}
\linenumbers

\numberwithin{equation}{chapter}

\usepackage[english]{babel}
\usepackage[babel]{csquotes}
\usepackage[doublespacing]{setspace}

\usepackage[
  detect-mode=true,
  detect-weight=true
  ]{siunitx} % get math/text mode from context

\setcounter{secnumdepth}{3} % number up to subsubsections

% --- biblatex with sorting, position, authoryear
\usepackage[
	backend=biber,
	style=authoryear-comp,
	giveninits=true,
	uniquename=init,
	uniquelist=false,
	hyperref=true,
	maxcitenames=2,
	maxbibnames=99,
	dashed=false,
	url=false,
	doi=true,
]{biblatex}

\renewcommand*{\nameyeardelim}{\addcomma\space} % comma bw author and year

\DeclareNameAlias{sortname}{family-given}
\renewbibmacro{in:}{%
  \ifentrytype{article}{}{%
  \printtext{\bibstring{in}\intitlepunct}}}

\bibliography{../../bibliography/phdt}

\usepackage{geometry}
\geometry{
  left=40mm, right=40mm,
  bindingoffset=0mm,
  top=40mm, bottom=40mm,
  heightrounded}

\usepackage{xcolor}
\definecolor{linkblue}{RGB}{0,51,153}
\definecolor{zgraylt}{RGB}{252,252,252}
\definecolor{zgraydk}{RGB}{200,200,200}
\definecolor{commentgray}{RGB}{95,95,95}
\definecolor{tablebackground}{gray}{0.95}
 % common color definitions

\raggedbottom % fine for drafts

\usepackage{hyperref}
\hypersetup{
  pdfpagelayout=SinglePage, % default
  pdfpagemode=UseOutlines, % default
  bookmarksopen, % default
  bookmarksopenlevel=2, % default;
  bookmarksdepth=3 % up to subsubsection (hidden in ToC)
  pdftitle={},
  pdfauthor={Alberto Pastorutti},
  pdfsubject={},
  pdfkeywords={},
  colorlinks=true,
  urlcolor=linkblue,
  linkcolor=linkblue,
  citecolor=linkblue}

% from tex.stackexchange.com/a/27107
% link on whole citation (author + year)

\DeclareFieldFormat{citehyperref}{%
  \DeclareFieldAlias{bibhyperref}{noformat}% Avoid nested links
  \bibhyperref{#1}}

\DeclareFieldFormat{textcitehyperref}{%
  \DeclareFieldAlias{bibhyperref}{noformat}% Avoid nested links
  \bibhyperref{%
    #1%
    \ifbool{cbx:parens}
      {\bibcloseparen\global\boolfalse{cbx:parens}}
      {}}}

\savebibmacro{cite}
\savebibmacro{textcite}

\renewbibmacro*{cite}{%
  \printtext[citehyperref]{%
    \restorebibmacro{cite}%
    \usebibmacro{cite}}}

\renewbibmacro*{textcite}{%
  \ifboolexpr{
    ( not test {\iffieldundef{prenote}} and
      test {\ifnumequal{\value{citecount}}{1}} )
    or
    ( not test {\iffieldundef{postnote}} and
      test {\ifnumequal{\value{citecount}}{\value{citetotal}}} )
  }
    {\DeclareFieldAlias{textcitehyperref}{noformat}}
    {}%
  \printtext[textcitehyperref]{%
    \restorebibmacro{textcite}%
    \usebibmacro{textcite}}}
 % link on whole citation (only on year otherwise)

\begin{document}

\setcounter{chapter}{2}
\chapter{The Earth crust as sensed by inverse modelling of the gravity field}

\setcounter{section}{1}
\section{Signal isolation: data reduction and uncertainty budget}

\subsection*{Summary}
In order to perform geophysical inversions on the static gravity field, a signal separation procedure is required, i.e.~the effect of an a-priori mass distribution must be removed, isolating an anomalous field \parencites[e.g.][]{Tenzer2012}{Sjoberg2013}[][and references therein]{Tenzer2009}.
The observed field stripped of the forward-modelled attraction of these `known masses' constitutes the input of inverse gravity modelling.
Any distribution of anomalous masses that result in the isolated field is a solution to the inverse problem.
Common reductions include the well-known terrain correction, i.e.~landmasses and bathymetry \parencites{Hinze2003}{Hinze2005}, which can be further refined by stripping the effect of subsurface density variations \parencite{Vajda2008}.

Uncertainties in the a-priori masses, such as unmodelled spatial variability in density, accumulate in the reductions and are propagated to the reduced data and to the inversion results.
Incomplete and inexact geological knowledge cannot be avoided, thus a perfect data reduction is unrealistic.
Therefore, an uncertainty-aware signal isolation process, i.e.~a process providing error estimates alongside its reduced gravity data, is a necessary step in estimating confidence of the inversion results, in statistical terms.
Error information is required in order to perform multi-observable integrated modelling \parencite[e.g.][]{Afonso2013}, comparisons \parencite[e.g.][]{Root2017}, or assimilation of geophysical models \parencites[e.g.~to assemble large scale maps from local studies, see][]{Tesauro2008}{Grad2009}{Molinari2011}.

The procedure and experiments described in this section were designed for local Moho inverse modelling, using global satellite-only gravity field models.
They were applied in the gravity-constrained thermal modelling strategy described in this thesis.
They include corrections for the terrain effect (topography, water, and ice), sediments and other intra-crustal density variations, far-field effects of crustal roots, and inhomogeneities in the lithospheric- and upper-mantle.
Particular attention was paid to ensure spectral consistency between the forward-modelled reductions and the input gravity models.

Uncertainty was quantified through Monte Carlo error propagation \parencite{Aster2018}, resulting in a population of randomly perturbed reduction fields (according to a set of parameter uncertainty assumptions) and of inverted crust-mantle boundary depths.
Results include the probability density function of these reductions and of the inverted results, from which derived descriptive statistics were computed (e.g.~the spatial distribution of a defined confidence interval).

% The adjective `anomalous' is commonly used to indicate deviations from an a-priori reference. In order to prevent confusion with the strict-sense `anomaly', a gravity functional, in this section the term \textit{residual} is used to refer to any gravity data after the signal isolation procedure.
\newpage

\subsection{Forward modelling of reductions to the gravity disturbance}
% teoria, definizioni, metodo SH e confronti con tesseroids



\subsection{Input a-priori mass distribution models}
% topografia, water, ice, litho, conversioni velocità

\subsection{Implementation}
% multiprocessing (subsection discutibile)

\subsection{Global reduction results}
% con valutazioni spettrali

\subsection{Reductions for regional crustal modelling}
% test locale, simile ad articolo, confronto con altro modello con incertezza

\subsection{Discussion}

% input this at the end of each chapter-draft

\printbibliography

