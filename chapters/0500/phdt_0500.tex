\chapter{Conclusions}
\label{c:Conclusions}
% citazione possibile qui, con \dictum[autore]{testo}

The research presented in this dissertation dealt with the application of satellite-only gravity models to modelling of the temperature distribution in the continental lithosphere and surface heat flow.
In the wider context of using gravity to perform inference on the structure of the planet interior, I have provided contributions in the following topics:
\begin{itemize}
    \item forward modelling of prior models of sub-surface density, pursuing signal isolation in gravity data;
    \item crustal-thickness-constrained geothermal modelling, aimed at using a result of gravity inversion to turn the sparse measurements of surface heat flow in a complete thermal volume, adding information at surface and in depth.
\end{itemize}

% intro alle conclusioni: motivazione (che poi riprenderò in introduzione)
Interest in addressing such a problem stems from the difference between regional scale gravity (e.g.~about~\SI{100}{\kilo \metre}) and surface heat flow measurements.
The former is an observable that can benefit from unparalleled global homogeneity in sampling and quality, thanks to satellite gravity missions.
The latter suffers from the well-known spatial sampling issues that affect all terrestrial measurements.

Consequently, when starting to enquiry if \emph{``gravity may fill-in the heat flow blanks''}, the main research question was formulated: \emph{``can we use gravity and thermal modelling to provide a physics-based interpolation of surface heat flow?''}.

The methodology developed here aimed at assessing this, starting from an a-priori experimental constrain: reliance on external data, other than satellite-gravity models and surface heat flow, should be kept at a reasonable minimum.
Therefore, as discussed in depth in chapter~\ref{c:ThermAppl}, the results are unavoidably affected by the well-known issues of non-uniqueness that are commonly addressed with multi-observable modelling \parencite{Afonso2013multiobsI}, such as partition between basal heat flow and crustal heat production.

However, even under the limiting conditions imposed, the results have provided a satisfactory consistency with current knowledge of the modelled study area, across the Trans-European Suture Zone -- a relatively well surveyed zone, compared to most of the Earth surface.
The proposed method could thus be scaled to surface heat flow fill-in in areas devoid of measurements, fully exploiting the potential of gravity models.

The methodology described in chapter~\ref{c:SigIs} and the associated data-reduction experiments provide an efficient framework for turning global mass distribution models in global gravity reductions, a fundamental step in the gravity inverse problems involved in solid-Earth geophysics.
The relief forward modelling method by \textcite{Wieczorek2007} was employed in a `layer-wise' configuration, based on the difference between the gravity field of the bottom and top boundaries defining a layer.
The proposed method competes with spatial-domain forward modelling, especially in at the comparatively low resolutions imposed by the current state-of-the-art both in satellite only gravity models and global compilations of subsurface data.

In the forward-modelling test I carried out, the input densities were modelled as contrasts against a reference density model.
Common solutions for this aspect involve the computation of average densities from the layers of the input model, e.g. of the whole crust \parencites{Tenzer2015}{Tenzer2019}.
Here, a novel solution was designed and implemented: the reference density model is defined in concentric shells of constant thickness.
Since each shell may intersect different layers, with large differences in layer-wise average densities, this strategy enabled the concurrent modelling of different layers referred to the same reference.
When average densities are compiled using complete shells, if necessary filling in the portion not included in the input model (e.g. the sub-lithospheric mantle) with a reference value, zero-average shell-wise corrections can be computed.

The obtained result is a whole-lithosphere reduction field, in spherical harmonics and synthesised global maps of gravity corrections.
It was applied to the combined satellite-only {GOCO06s} \parencite{Kvas_GOCO06s_dataset}, for which the residual gravity disturbances are presented, after reduction.
Spectral domain comparisons were performed between the modelled masses and the input gravity model.
Among their results, the effective bandwidth of the input density model was assessed: there is a sudden decay in degree cross-correlations between SH degree \num{60} and \num{110} (half-wavelengths of \num{3} to \num{1.6} arc degree).

Pursuing the objective of estimating what is the error introduced by data reductions, a Monte Carlo uncertainty propagation scheme was set up and tested with a simplified error model for the input data.
It may be scaled to more complex error models.

The thermal application of chapter~\ref{c:ThermAppl} made extensive use of the method and findings of the data reduction strategy, chiefly in two steps of the Moho signal isolation procedure:
(1) in modelling the far-field `isostatic effect', to remove the long-wavelength component that affects gravity data from GGM after correcting them with global terrain effect models; (2) in harmonising the more detailed regional sediment thickness model that was utilised inside the study area \parencite[{EuCRUST},][]{Tesauro2008} with a global low-resolution model \parencite[{LITHO1.0},][]{Pasyanos2014}, in order to perform spectral-domain filtering on a global spherical harmonics model of sediments correction.

% qui passo a termico
After the gravity data reduction step, the thermal strategy relies on inversion of the isolated Moho signal.
Thickness of the consolidated crust is the quantity that enables the fill-in of the areas devoid of surface heat flow measurements, after fitting the radioactive heat production in cells with measurements.
The obtained Moho images a crustal morphology which is comparable with other models in the regions, adopted as benchmarks.
Gravity based method have a well proven record in retrieval of crustal thickness, showing their most promising results in integrated inverse modelling strategies \parencites[e.g.][]{Eshagh2011}{Reguzzoni2015}, combining data from different geophysical observables.
Here, by design, the performance of a satellite-only approach was assessed, relying as less as reasonably possible on other sources.

% dettagli modello termico
A 3D finite-difference thermal forward operator for solving the steady state heat equation has been developed for the purposes of this research.
It allows a radioactive heat production inversion technique, which progresses iteratively through a succession of 1D approximations and subsequent substitution of the parameter sought for.
It enabled the production of a complete volume of temperature and thermal parameters in the study area, resulting in a gravity-controlled, physics-based, fill-in of the no-data cells -- both at surface and in depth.

% qui: "IN SUMMARY..."
% dai finding fatti in questa tesi, propongo che
The findings of the experiments and applications described in this dissertation suggest that:
\begin{itemize}
    \item aaa % qui magari messi come: finding -> suggest that, based on finding
    \item % prendo dalla lista delle conclusioni del termico!
\end{itemize}

% seguente: INTEGRATO NELL'ITEMIZE QUI SOPRA
% inoltre [further work]
% sottolineando che queste non sono idee a caso
% ma arrivano dai finding in questa tesi

% global per vincolare bassi gradi - ma qual è l'effetto locale dell'errore?

%%% TEMP

and when compared with another crust-constrained approach
relationship

% issues with that
% gravity small formal error, largest part due to reductions
% heat flow not very sensitive


to 
addressed with geostatistical methods
In summary:
could a method
could this method lean
therefore
provide
carries an associated uncertainty

Existing literature on integrated geophysical modelling

surface heat flow on its own provides a weak constraint
for the inverse modelling of lithospheric structures
and the distribution of temperature at depth

most of

on the other hand, the surface heat flow spatial distribution
controlled by structures at depth

allows for
quality of forward modelling
fit for purpose (e.g. geothermal application, petroleum system modelling)
% citazioni (es scheck wanderoth)
% e poi accenno a strength, reologia

% passo a questione: fitting of RHP
The thermal modelling strategy of chapter~\ref{c:ThermAppl} was thus devised, relying on a lean forward operator (appendix~\ref{c:ThermModel}).

Uncertainty in crustal thickness
% accenno che Kaban2004 called for cose del genere, in ambito gravità

% fine "introduzione delle conclusioni", passo a ciò che ho fatto

Starting from these premises

% risultati