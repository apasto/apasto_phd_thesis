\chapter{Conclusions and outlook}
\label{c:Conclusions}
% citazione possibile qui, con \dictum[autore]{testo}

The research presented in this dissertation dealt with the application of satellite-only gravity models to modelling of the temperature distribution in the continental lithosphere and surface heat flow.
In the wider context of using gravity to perform inference on the structure of the planet interior, I have provided contributions in the following topics:
\begin{itemize}
    \item forward modelling of prior models of sub-surface density, pursuing signal isolation in gravity data;
    \item crustal-thickness-constrained geothermal modelling, aimed at using a result of gravity inversion to turn the sparse measurements of surface heat flow in a complete thermal volume, adding information at surface and in depth.
\end{itemize}

% intro alle conclusioni: motivazione (che poi riprenderò in introduzione)
Interest in addressing such a problem stems from the difference between regional scale gravity (e.g.~about~\SI{100}{\kilo \metre}) and surface heat flow measurements.
The former is an observable that can benefit from unparalleled global homogeneity in sampling and quality, thanks to satellite gravity missions.
The latter suffers from the well-known spatial sampling issues that affect all terrestrial measurements.

Consequently, when starting to enquiry if \emph{``gravity may fill-in the heat flow blanks''}, the main research question was formulated: \emph{``can we use gravity and thermal modelling to provide a physics-based interpolation of surface heat flow?''}.

The methodology developed here aimed at assessing this, starting from an a-priori experimental constrain: reliance on external data, other than satellite-gravity models and surface heat flow, should be kept at a reasonable minimum.
Therefore, as discussed in depth in chapter~\ref{c:ThermAppl}, the results are unavoidably affected by the well-known issues of non-uniqueness that are commonly addressed with multi-observable modelling \parencite{Afonso2013multiobsI}, such as partition between basal heat flow and crustal heat production.

However, even under the limiting conditions imposed, the results have provided a satisfactory consistency with current knowledge of the modelled study area, across the Trans-European Suture Zone -- a relatively well surveyed zone, compared to most of the Earth surface.
The proposed method could thus be scaled to surface heat flow fill-in in areas devoid of measurements, fully exploiting the potential of gravity models.

The methodology described in chapter~\ref{c:SigIs} and the associated data-reduction experiments provide an efficient framework for turning global mass distribution models in global gravity reductions, a fundamental step in the gravity inverse problems involved in solid-Earth geophysics.
The relief forward modelling method by \textcite{Wieczorek2007} was employed in a `layer-wise' configuration, based on the difference between the gravity field of the bottom and top boundaries defining a layer.
The proposed method competes with spatial-domain forward modelling, especially in at the comparatively low resolutions imposed by the current state-of-the-art both in satellite only gravity models and global compilations of subsurface data.

In the forward-modelling test I carried out, the input densities were modelled as contrasts against a reference density model.
Common solutions for this aspect involve the computation of average densities from the layers of the input model, e.g. of the whole crust \parencites{Tenzer2015}{Tenzer2019}.
Here, a novel solution was designed and implemented: the reference density model is defined in concentric shells of constant thickness.
Since each shell may intersect different layers, with large differences in layer-wise average densities, this strategy enabled the concurrent modelling of different layers referred to the same reference.
When average densities are compiled using complete shells, if necessary filling in the portion not included in the input model (e.g. the sub-lithospheric mantle) with a reference value, zero-average shell-wise corrections can be computed.

The obtained result is a whole-lithosphere reduction field, in spherical harmonics and synthesised global maps of gravity corrections.
It was applied to the combined satellite-only {GOCO06s} \parencite{Kvas_GOCO06s_dataset}, for which the residual gravity disturbances are presented, after reduction.
Spectral domain comparisons were performed between the modelled masses and the input gravity model.
Among their results, the effective bandwidth of the input density model was assessed: there is a sudden decay in degree cross-correlations between SH degree \num{60} and \num{110} (half-wavelengths of \num{3} to \num{1.6} arc degree).

Pursuing the objective of estimating what is the error introduced by data reductions, a Monte Carlo uncertainty propagation scheme was set up and tested with a simplified error model for the input data.
It may be scaled to more complex error models.

The thermal application of chapter~\ref{c:ThermAppl} made extensive use of the method and findings of the data reduction strategy, chiefly in two steps of the Moho signal isolation procedure:
(1) in modelling the far-field `isostatic effect', to remove the long-wavelength component that affects gravity data from GGM after correcting them with global terrain effect models; (2) in harmonising the more detailed regional sediment thickness model that was utilised inside the study area \parencite[{EuCRUST},][]{Tesauro2008} with a global low-resolution model \parencite[{LITHO1.0},][]{Pasyanos2014}, in order to perform spectral-domain filtering on a global spherical harmonics model of sediments correction.

% qui passo a termico
After the gravity data reduction step, the thermal strategy relies on inversion of the isolated Moho signal.
Thickness of the consolidated crust is the quantity that enables the fill-in of the areas devoid of surface heat flow measurements, after fitting the radioactive heat production in cells with measurements.
The obtained Moho images a crustal morphology which is comparable with other models in the regions, adopted as benchmarks.
Gravity based methods have a well proven record in retrieval of crustal thickness, showing their most promising results in integrated inverse modelling strategies \parencites[e.g.][]{Eshagh2011}{Reguzzoni2015}, combining data from different geophysical observables.
Here, by design, the performance of a satellite-only approach was assessed, relying as less as reasonably possible on other sources.

% dettagli modello termico
A 3D finite-difference thermal forward operator for solving the steady state heat equation has been developed for the purposes of this research.
It allows a radioactive heat production inversion technique, which progresses iteratively through a succession of 1D approximations and subsequent substitution of the parameter sought for.
It enabled the production of a complete volume of temperature and thermal parameters in the study area, resulting in a gravity-controlled, physics-based, fill-in of the no-data cells -- both at surface and in depth.
\\

% qui: "IN SUMMARY..."
% dai finding fatti in questa tesi, propongo che
In summary, with the findings of the experiments and applications carried out in this dissertation, I have shown that:
\begin{enumerate}
    \item \textsf{\textbf{Thermal modelling strategy:}}
    a radioactive heat production fitting strategy, constrained by uniformly-sampled crustal thickness and sparsely-sampled surface heat flow, can provide added thermal information on areas lacking direct measurements, either because they are not yet available (surface heat flow) or are logistically not feasible (temperature at depth).
    Estimates are comparable with the results of petrophysical relationships.
    In terms of open issues, the tests demonstrated how uncertainty in crustal thickness is more critical the higher the estimated basal heat flow ($Q_{M}$). In addition, the ambiguity in $Q_{M}$ separation is affected by the potentially unstable behaviour of temperature-dependent thermal conductivity.
    \begin{description}
        \item[\quad Further development] Multivariate and probabilistic inversion seems justified even for an inverse problem constrained only by two observables (gravity and heat flow). Topography, another observable not suffering from sampling issues (at this scale) could provide added informations in assessing the basal heat flow: the effective elastic thickness.
        In addition, the predictive performance of the physics-based fill-in strategy could be quantitatively assessed against the simple interpolation of surface heat flow values, resorting to cross-validation methods \parencite[e.g.][, chapter~1]{Bishop2006ML}
    \end{description}

    \item \textsf{\textbf{Moho inversion for regional thermal modelling:}}
    A well-tested procedure was employed to invert a Moho depth from the isolated gravity signal, in local cartesian coordinates.
    The performance of this crustal thickness estimate are comparable with those obtained by using an integrated seismic based Moho: it is successful in resolving the deep structures controlling the shallow thermal regime and in providing a uniformly-sampled quantity for the RHP-fitting strategy.
    Results are significantly better than those provided by a checker board Moho test, with higher sensitivity in cells of higher heat flow.
    \begin{description}
        \item[\quad Further development] The effect of temperature on density, and then on Moho inversion, was assessed using the output of the forward thermal model: it results in variations of \SI{-12.2}{\percent} to \SI[retain-explicit-plus]{+14.8}{\percent} respect to the Moho inversion that disregarded the thermal effect. While this is deemed an acceptable uncertainty, it would make sense to not discard the temperature information that this modelling method provides. This is also suggested by the existing research on thermal effects on crustal gravity modelling have already provided  results \parencite[e.g.][]{Bagherbandi2017thermal}.
    \end{description}

    \item \textsf{\textbf{Satellite-only gravity for thermal modelling:}}
    \begin{itemize}
        \item The suitability of gravity as an independent constraint in thermal modelling was demonstrated in this thesis with a proof-of-concept regional application.
        The continental lithosphere poses lots of issues due to local heterogeneities that superimpose over the necessary simplification of general models.
        Reality deviates from a simple two-layer crust-mantle model, and thermal parameters are no exception in this -- the vertical distribution of radioactive elements being only a first-order approximation \parencites{Jaupart2003}{Alessio2018deepRHP} and active tectonics hindering the applicability of steady-state modelling.
        The thermal strategy mentioned above was designed to overcome these issues, at least in part.
        A compromise between uniform coverage and resolution is required -- a direct consequence of the band-limited input data provided by the input gravity data.
        In addition, uncertainties in the a-priori data reductions are one to two orders of magnitude larger than the \si{mGal}-level cumulative formal errors of the GGMs.
        Nevertheless, data from a state-of-the-art satellite-only GGM have been shown suitable for this purpose.
        \item  The thermal application included a bare-minimum integration with external data sources: a sedimentary reduction using a basement depth model (local and global), a low-resolution Moho density contrast model, removal of the distant isostatic effect using a global crustal model, and a thermal lithospheric thickness provided by a surface wave inversion model.
        Most of these data sets are available at least at low resolution globally, albeit assessing their data quality can be non trivial.
        The data-reduction forward modelling presented in this thesis, in addition to the computation of global corrections and uncertainty propagation, proposes spectral comparison as a tool to assess coherency in effective bandwidth between input gravity models and reductions.
    \end{itemize}
    \begin{description}
        \item[\quad Further development]
            Scaling the thermal modelling framework to larger extents, e.g. on all the continental crust using a moving-window approach, is its natural further step.
            While unavailability of terrestrial data is not easily overcome, uncertainty-aware modelling can at least provide ``error bars'' on the output of a modelling strategy.
            Propagation of uncertainties in the reductions to the final thermal estimate is surely an aspect that warrants further investigation.
            The lean modelling tools that the modelling work flow presented here is composed of throughout (from gravity reduction to thermal modelling) could easily allow such a development.
    \end{description}
\end{enumerate}

