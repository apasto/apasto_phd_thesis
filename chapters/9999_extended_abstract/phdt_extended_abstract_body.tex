% Global models of the Earth gravitational fields

% suitable for

% structure of the lithosphere

% ...
% recent developments
% 3d modelling of earth lithosphere
% static snapshot
% insight into dynamics
% (e.g. back-propagation ...)

The heat transport across the Earth surface, arising from its cooling interiors, senses a wide assortment of phenomena, from near-surface paleoclimatic perturbations \parencite{Majorowicz2011} to the driving forces of plate tectonics and mantle dynamics \parencite{Cooper2017}.
Surface heat flow is sensed with in situ measurements of temperature and conductivity, usually in boreholes.
This implies that heat flow is a costly observable, highly depending on logistic and financial constraints, on previous exploration interest and on availability of data.
The collection and maintenance of a global database is an ongoing effort, spanning multiple decades \parencites{Lee1965}{Hasterok2008}.

Different components superimpose on the surface heat flow signal.
Over continents, at regional scales (from \SI{100}{\kilo \metre} onwards), most of the static signal can be explained by a partition between a `basal heat flow' (through the crust-mantle boundary) and a crustal radiogenic component ---i.e. the result of radioactive decay occurring in the crust \parencite{Jaupart2016}.
The generation of continental crust through magmatic differentiation (partial melting and crystallization) significantly enriched the upper portions of the lithosphere with radioactive elements, among which uranium, thorium, and potassium account for almost the total heat production.
These are highly mobile trace elements, and their concentration does not directly affect geophysical quantities by a significant amount \parencite{Hasterok2017_mis}.

Therefore, radioactive heat production is a highly uncertain physical property.
However, other petrogenetic processes concur in affecting sensible quantities (e.g.~seismic velocities, density).
Empirical relationships from velocity and/or density to heat production have developed and tested satisfactorily \parencites{Rudnick2003}{Vila2010}{Hasterok2017_ign}.
In addition, multi-observable modelling \parencites[e.g.][]{Mather2018}{Afonso2019} is regularly employed to improve the understanding on the subsurface distribution of temperature and thermal parameters, exploiting their effect on directly sensed quantities (e.g.~topography, gravity, seismic, and electromagnetic observations).

The gravity field is particularly sensible to variations in crustal thickness, arising from the significant density contrast at the Moho (a global average of \SI{485}{\kilo \gram \per \cubic \metre} according to \cite{Tenzer2012}, of \num{448}~$\pm$~\SI{187}{\kilo \gram \per \cubic \metre} according to \cite{Sjoberg2011}, to name two).
Compared with the problematically sampled heat flow, gravity data benefits from a much more uniform coverage, instead.
While high resolution terrestrial data suffers from the same issues that affect heat flow (and )
It is therefore tempting to assess how 

make us of

if a % relazione soddisfacente con gravità

compiling requires "fill in"
% spatial regularisation?
lithosphere component
short scale
backstripping


Satellite gravity missions provide % ... solito
% uniform sampling


